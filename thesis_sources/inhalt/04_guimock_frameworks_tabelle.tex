\renewcommand{\tabularxcolumn}[1]{>{\raggedright\arraybackslash}p{#1}}
\newcolumntype{s}{>{\hsize=.84\hsize}X}
% \newcolumntype{l}{>{\hsize=.8\hsize}X}

% \begin{table}[htbp!]
%     \begin{center}
%         \caption{Klassifikationsergebnisse Kategorie \glqq Eingabedaten\grqq}
%         \bigskip

        \pgfplotstabletypeset[
            reset styles,
            debug=true,
            string type,
            header=false,
            col sep=semicolon,
            row sep=newline,
            column type=,
            skip first n=1,
            begin table={
              \begingroup\renewcommand{\arraystretch}{1.5}
              \begin{xltabular}{\textwidth}{XXXXXX}
                \caption{Untersuchte GUI-Bibliotheken und -Frameworks (Stand: 30.06.2021)} \label{tab:frameworks} \\
              \toprule
              \textbf{Name} & \textbf{Stand} & \textbf{Renderer / basiert auf / Schnittstelle} & \textbf{Anmerkung} & \textbf{Plattform} \\
              \midrule
              \endhead
            },
            end table={\bottomrule \end{xltabular}\endgroup},
            every head row/.style={ output empty row },  % suppress printing head row (numbers)
            % every head row/.style={after row=\midrule},
            % display columns/0/.style={string type,column name={\textbf{Name}}
            % },
            % display columns/1/.style={string type,column name={\textbf{Stand}}},
            % display columns/2/.style={string type,column name={\textbf{Renderer / basiert auf / Schnittstelle}}},
            % display columns/3/.style={string type,column name={\textbf{Anmerkung}}},
            % display columns/4/.style={string type,column name={\textbf{Plattform}}},
          ]{tabellen/FrameworksTabelle.csv}
%         \label{table1}
%     \end{center}
% \end{table}
