% \chapter{Evaluierung}
% Nachdem die Implementierungsarbeiten abgeschlossen sind, werden einige Experimente und Messungen durchgeführt. Im Vordergrund steht hierbei die Evaluierung der Autoencoder-Architekturen, d.h. wie gut diese jeweils abschneiden und wo die Stärken und Schwächen liegen. Darüber hinaus findet auch eine allgemeine Evaluierung der Stärken und Schwächen von Autoencodern im konkreten Anwendungsfall statt. Die Experimente zielen auf die Beantwortung der in der Masterarbeit zu beantwortenden Forschungsfragen, wie sie in der GQM-Analyse in Abschnitt NO GQM beschrieben werden.
% \todo{proposal wg GQM referenzieren?}
% Die Experimente laufen dabei so ab, dass Autoencoder mit generierten Daten trainiert werden. Anschließend werden mehrere Metriken berechnet, aus denen sich bestimmte Schlüsse hinsichtlich der Forschungsfrage ableiten lassen. Im oben genannten Abschnitt über die GQM-Analyse werden zu jeder zentralen Forschungsfrage die Metriken beschrieben, die die Experimente charakterisieren.
% % \begin{itemize}
% %     \item Was für Experimente werden durchgeführt?
% %     \item Was ist wichtig?
% %     \begin{itemize}
% %         \item Performance
% %         \item Welche Autoencoder können gut (=gute Ergebnisse) vom am FZI entwickelten CTRNN verwendet werden?
% %     \end{itemize}
% % \end{itemize}
